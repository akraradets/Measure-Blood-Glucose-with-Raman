\setlength{\footskip}{8mm}

\chapter{TITLE}

Write your introductory paragraph/s to give an overview of the chapter (except for Chapter 1). Limit this section to two paragraphs. Follow the appropriate structure of writing paragraphs. Paragraphs should have at least four sentences (8 lines). Paragraphs with more than 6 sentences (12 lines) must be split into two paragraphs.  Maintain one blank line between paragraphs.

\begin{figure}
\caption{Doge.}
\centerline{\includegraphics[width=3in]{figures/doge.jpeg}}
\label{fig:doge}
\small{\textit{Note.} Additional notes goes here.}
\end{figure}

\section{Heading, Level 2}

This section presents some guidelines on how to create and format tables and figures following the APA Style with some examples. Every table and figure should serve a purpose. A table or figure can be referred to in the text by its number [e.g., As shown in Table \ref{tab:dense}…, as can be seen in the results of the testing (see Figure \ref{fig:doge})].  Avoid writing “the table above” or “the figure below” as the position of a figure or table might change during the writing process.

Tables and figures can be generated in different ways using many programs.  Table \ref{tab:dense} presents the format of a table following the APA style. Align all tables and figures with the left margin and place a table or figure after a paragraph where it is first mentioned. Separate the paragraph and the table or figure title by a double-spaced blank line. Titles should be brief, clear, and explanatory. 

Repeat the column headings on the second page of the table (see Table \ref{tab:dense}). Separate this paragraph from the table by a double-spaced line. Tables and figures can be placed at the start or end of a page. Fit the table or figure between the margins and in one page.

As there is very little space left for the table on this page, present the table on the next page.  You can add more content in this section.  The description should be as close to the table or figure as possible.
(There should be one blank double-spaced line between the last line of the paragraph and the table or figure number, and between the table / figure number and the title.)

You can cite stuff in references.bib like this \citep{doge}. $y$ is as follows.

\begin{equation}
    y = mx+b
\end{equation}

\section{Heading, Level 2}

Add a short introductory sentence/s here.

\subsection{Heading, Level 3}

Start your paragraph here. Table 2.2 presents a sample of a qualitative table with variable descriptions. Separate the paragraph and the table or figure title by a double-spaced blank line. Titles should be brief, clear, and explanatory.  Check the Language Center website for more examples.

\subsection{Heading, Level 3}

(1 space between the last line of this section and the next Level 2 heading)

\section{Heading, Level 2}
As for figures, the figure title should also be written in italics below the figure number (in bold) separated by a double-spaced blank line as shown in Figure 2.1.  The size and density of the elements in a figure must be considered when deciding on the font size and spacing.   Continue with the paragraph here.

Continue with the paragraph here.  The table or figure should be as close to the description as possible or when it is first mentioned. Fit the tables and figures between the margins.
(There should be one blank double-spaced line between the previous paragraph and the figure number, and between the figure number and the title.)

\begin{table}[]
\caption{An example table in latex.}
\begin{center}
\begin{tabular}{l l}
\hline
    Methods & Metric\\ \hline
Method A      & 153.3                \\ 
Method B & 2.4                  \\ \hline
\end{tabular}
\label{tab:dense}
\end{center}
\small{\textit{Note.} Add notes here.}
\end{table}
