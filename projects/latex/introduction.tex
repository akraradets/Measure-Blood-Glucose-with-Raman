\chapter{INTRODUCTION} 

\section{Background of the Study}

% Measures of Accuracy for Continuous Glucose Monitoring and Blood Glucose Monitoring Devices

Continuous glucose monitoring (CGM) systems have been recognized as a key factor for effective glycemic control of diabetic patients \citep{continuous2021}.
CGM refers to automatic, continuous (real-time or periodic) monitoring of users' glucose through invasive, minimally invasive (e.g., small incisions), and non-invasive means.
To date, the acceptable and commercialized CGM approach is through sensor implantation, but it requires lengthy calibrations, sometimes unreliable and minimally-invasive \citep{CGMminimalinvasive}.

Non-invasive techniques through analyte (e.g., glucose solution, interstitial fluid (ISF)) analysis have attracted much interest.
Optical-based methods were proven to yield superior results, achieving strong selectivity of glucose fingerprints on complex analytes such as blood \citep{opticalBest}.
Among the optical-based methods (e.g., far infrared to fluorescence spectroscopy), Raman spectroscopy appears promising due to its insensitivity to water (e.g., as compared to near-infrared) and its ability to accurately measure glucose quantitatively and transcutaneously \citep{directGlucose}.
Anyhow, Raman spectroscopy comes with challenge, as is often confounded with fluorescence artifacts, but of which is commonly countered by adjusting the laser intensity or measuring times.  
The rise of Raman spectroscopy is also timely due to its recent advancement of laser technology~\citep{horibahistory}.

The use of Raman spectroscopy for measuring blood glucose can be dated back as far as 2005.
~\cite{forearm2005} found a strong association ($R^2 = 0.83$) on Raman spectra between crystallized glucose and ISF measured at human forearm.
~\cite{solutionGlucose} confirmed a strong association ($R^2 = 0.91$) on Raman spectra between concentration on glucose solution and ISF measured at mouse ear.
~\cite{directGlucose} demonstrated a new approach to extract glucose scattering by subtracting two Raman signals from two different time points as a direct measurement of glucose in blood.
In addition, they validated that to reliably measure the glucose concentration in blood, the glucose peak ($1125 \text{cm}^{-1}$) should be normalized with protein and lipid peak ($1450 \text{cm}^{-1}$).
They achieved an $R^2 = 0.91$ between actual glycemic and predicted glycemic using Raman spectra measured from a pig's ear.
The measuring site is another important variable. Forearm \citep{forearm2005, forearm2014}, thenar \citep{glucobeam}, and nail fold \citep{ramanNailFold2019} have been chosen as promising measurement sites.
While~\cite{sitecompare} indicates that the forearm is the most effective site when compared to the wrist and index finger, it remains unclear which site is the best due to varying equipment, parameters, and methodology (e.g., how to preprocess) across the papers.
Due to the portability of wearables, there has also been some very recent attempt to deploy Raman spectroscopy on wearable (e.g., smartwatch) commercially~\cite{ramanGlucoseWatch} but the research is still in its infancy.

This research aims to build on previous work by 
(1) confirming the use of Raman scatterings for measuring blood glucose, 
(2) comparing models, and 
(3) developing the first wearable (continuous, non-invasive, pervasive) Raman-based self-monitoring blood glucose (SMBG) system, primarily for daily users for widespread use.
The accuracy of the glycemic measurement should be comparable to that of the well-respected SpO2 wearable sensor (Apple Watch 6) of ($R^2 = 0.81$, $p < 0.001$)~\citep{applewatchaccuracy}.
This justification was made because Raman spectroscopy was demonstrated to have at most 90\% association with blood glucose. 
In addition, body movements may potentially confound the measurements, so it is advisable to set the objective for daily users rather than for clinical use.


\section{Statement of the Problem}

Due to methodological variations, it is challenging to compare previous findings.
For instance, five measuring sites of the human body have been reported with success. 
However, only the wrist, fingertip, and forearm are being compared with the same methodology \citep{sitecompare}.
The nail fold is a promising measuring site because its thin epidermis layer makes it easier for the laser to reach the microvessel in the dermis \citep{ramanNailFold2019}.
And thenar has been chosen as a measuring site for a portable Raman-based SMBG device \citep{glucobeam}.
Because all the work employs statistical-based analysis using the full spectrum, it can be argued that the high accuracy of glucose prediction may be due to the contribution of correlated artifacts signal rather than the glucose signal itself \citep{directGlucose}.
We can determine which site produces the strongest correlation between glucose fingerprint and glycemic using the direct measurement of glucose technique \citep{directGlucose}.

Data preprocessing and modeling are needed to be compared. 
As mentioned above, one method that has been employed widely is using statistical-based analysis using the full spectrum.
Another method is to normalize the recorded spectra with either hemoglobin peak (1549 $\text{cm}^{-1}$) \citep{solutionGlucose} or protein and lipid peak (1450 $\text{cm}^{-1}$) \citep{directGlucose}.
With the normalization method, a smaller set of spectra can be selected for modeling.
In addition, this method has been showed to produce a feature the has a linear relationship with glycemic \citep{solutionGlucose,directGlucose}.
However, the method has never been employed together with human subjects.


\section{Objectives}

Our objective is to develop a wearable SMBG.
To achieve this, we separate the project into four studies.

\subsection{Study 1: Confirming the parameters}

\textbf{Objective}: To study the measuring site and times.\\
\textbf{Independent Variables}: 
\begin{enumerate}
    \item Measuring time
    \item Measuring Site
    \begin{enumerate}
        \item Wrist
        \item Forearm
        \item Index fingertip
        \item Index nail fold
    \end{enumerate}
\end{enumerate}

\begin{sloppypar}
    \textbf{Dependent Variables}: Spectra peaks around 1125 $\text{cm}^{-1}$ with the lowest total measuring time.\\
    \textbf{Outcome}: Confirm the suited measuring site and times.
\end{sloppypar}


\subsection{Study 2: Raman scattering of blood glucose study}\label{intro-s2}

\textbf{Objective}: Study Raman scattering of blood glucose and build a model to predict the glycemic for a wearable device.\\
\textbf{Independent Variables}: Raman scattering of blood\\
\textbf{Dependent Variables}: Glycemic\\
\textbf{Outcome}: The model that results in glycemic prediction correlation $R^2 > 0.8$ with actual glycemic, and resource usage.

\subsection{Study 3: Designing and developing wearable blood glucose device}

\textbf{Objective}: Design and develop a prototype of a wearable SMBG.\\
\textbf{Outcome}: A prototype.


\subsection{Study 4: Device Evaluation}

\textbf{Objective}: To evaluate the prototype, we redo Section~\ref{intro-s2} experiment with our prototype.\\
\textbf{Independent Variables}: Raman scattering of blood\\
\textbf{Dependent Variables}: Glycemic\\
\textbf{Outcome}: Prototype achieves glycemic prediction correlation $R^2 > 0.8$ with actual glycemic.


\section{Organization of the Study}

The document is organized as following. Chapter~\ref{literatureReview} as Literature Review and Chapter~\ref{methodology} as Methodology.