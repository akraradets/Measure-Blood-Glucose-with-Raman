\chapter{INTRODUCTION} 

\section{Background of the Study}

Diabetes is now in the top three leading causes of death and disability in the United States \citep{cdc}. 
\cite{who} states that there are 422 million people who are affected by diabetes throughout the globe. 
This number has been rising for the past three decades and there is a global agreement to stop this trend by 2025.
Although there is no cure for this condition, controlling glycemic can help the patient to live 5 to 8 years longer \citep{controlGlycemic}.



\section{Statement of the Problem}

\begin{sloppypar}
However, today glucose meter requires a blood sample to be drawn by pricking the patient's finger \citep{todayGlucoseMeter}.
This invasive method causes both pain and risk of infection to the patients. Therefore, glycemic is not measured as often as recommended \citep{NIR2017, continuous2008, continuous2018}.
Therefore, we want to have a non-invasive self-monitoring blood glucose (SMBG) method which will enable continuous measurement.
\end{sloppypar}

A recent study has shown that Raman Spectroscopy is a viable technique to measure glycemic. 
By selecting the measuring site and employing the machine learning technique, the root-mean-square error (RMSE) can be as low as 0.26601 mmol/L \citep{ramanNailFold2019}. 
As a result, Raman Spectroscopy is selected as a measuring technique.

\section{Objectives}

Our objective is to develop a portable self-monitoring blood glucose meter. To achieve this, we separate the project into three studies.

\subsection*{Study 1: Confirm the parameters}

1. Reproduce the 
Introduce your main objective first in one sentence, followed by a bulleted list of specific objectives, left aligned.
1.	Follow a 0.5-inch Tab setting.
2.	Add more here.

\section{Organization of the Study}
Start your paragraph here.