\phantomsection
\addcontentsline{toc}{chapter}{\bf ABSTRACT}

\begin{center}
    \large{\bf ABSTRACT}
\end{center}

% Use this to fix the "/hbox is too wide" 
% https://latexref.xyz/sloppypar.html
\begin{sloppypar}
% 1. Background: Very specific background; hint the problem.
Continuous glucose monitoring (CGM) systems have been identified as a crucial component of successful glycemic management in diabetic patients \citep{continuous2021}.
Currently, the commercial CGM approach involves implanting a sensor \citep{CGMminimalinvasive}.
Its minimally-invasive nature prevents the pervasiveness of monitoring glucose.
The non-invasive approach such as Raman spectroscopy has been studied as a means to measure glycemic in vivo.
Thus, wearable (continuous, non-invasive, pervasive) self-monitoring blood glucose (SMBG) is possible.
% 2. Problem: Very very measurable problem; start with a signal word like “However”, “Anyhow”, or “Despite”.
However, wearable has its limitation and previous works do not study the aspect of this usage. 
% 3. Solution: Use verb wisely; explore/investigate/develop/compare.
Thus, a comprehensive comparison between measuring sites, measuring schemes, preprocessing techniques, and models is studied here.
Furthermore, a prototype of wearable SMBG is developed and evaluated.
% 4. Key finding (2 - 3 sentences): Summarize ONLY the key findings - it means interesting findings.
We found that the wrist is the best site to measure glycemic. 
When use normalized 1125 $\text{cm}^{-1}$, it achieved $R^2 = 0.9$ with blood glucose.
Our prototype achieves correlation with blood glucose over $R^2 = 0.8$ comparable to reputable wearable SpO2 sensors in the market. 
% 5. Contributions: Why this is important to be solved; what impact it can bring.
Our results contribute to (1) the best site to measure glycemic in the human and measuring scheme, 
(2) effective preprocessing technique and model for predicting the glycemic,
(3) prototyping the wearable SMBG using Raman spectroscopy.
\end{sloppypar}
