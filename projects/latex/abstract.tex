\phantomsection
\addcontentsline{toc}{chapter}{\bf ABSTRACT}

\begin{center}
    \large{\bf ABSTRACT}
\end{center}

% Use this to fix the "/hbox is too wide" 
% https://latexref.xyz/sloppypar.html
\begin{sloppypar}
% 1. Background: Very specific background; hint the problem.
Self-monitoring of blood glucose (SMBG) is essential for diabetics to monitor their glycemia (the concentration of sugar or glucose in the blood).
By measuring glycemia continuously, the treatment and patient lifestyle can be adjusted according to the report. 
The solution would be a wearable device that collects blood glucose continuously throughout the day.
% 2. Problem: Very very measurable problem; start with a signal word like “However”, “Anyhow”, or “Despite”.
However, building a complete non-invasive one is challenging. 
Multiple techniques employing non-optical and optical methods have been studied. 
One technique that showing to be promising for this task is Raman Spectroscopy due to its low sensitivity to water and temperature, and great specificity. 
% 3. Solution: Use verb wisely; explore/investigate/develop/compare.
Fortunately, researchers have shown that glycemia can be measured by observing the light scattering such as the Raman Spectroscopy technique. 
With this method, it is possible to develop portable blood glucose meter equipment that is not only non-invasive but also a continuous method.
% 4. Key finding (2 - 3 sentences): Summarize ONLY the key findings - it means interesting findings.
Our portable equipment has an accuracy comparable to the clinical glucose meter of choice with an error of less than $\pm$ 0.83 mmol/L.
The portable aspect enables continuous sampling throughout the day. 
% 5. Contributions: Why this is important to be solved; what impact it can bring.
With our equipment, continuous non-invasive SMBG is possible. 
Not only this will help to improve the treatment plan for diabetics but also enables individuals to monitor their glycemia which in turn may improve their diet selection.
\end{sloppypar}
